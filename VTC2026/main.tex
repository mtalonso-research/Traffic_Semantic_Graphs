\documentclass[conference]{IEEEtran}
\IEEEoverridecommandlockouts
% The preceding line is only needed to identify funding in the first footnote. If that is unneeded, please comment it out.
\usepackage{cite}
\usepackage{amsmath,amssymb,amsfonts}
\usepackage{algorithmic}
\usepackage{graphicx}
\usepackage{textcomp}
\usepackage{xcolor}
%\usepackage{mathcomSTEv4}
\usepackage{mathtools}
\newcommand*{\Break}{\textbf{Break}}
\newcommand{\na}{\mathsf{n \backslash a}}
\newtheorem{proposition}{Proposition}
\newtheorem{lemma}{Lemma}
\newtheorem{corollary}{Corollary}
\usepackage{array}
\usepackage{color}
\usepackage{bm}
\usepackage{mathrsfs}
\usepackage{algorithmicx}
\usepackage{algorithm}
\usepackage[noend]{algpseudocode}
\usepackage{epsf}
\usepackage{xpatch}
\usepackage{setspace} 
\usepackage{enumitem,kantlipsum}
\usepackage{multirow,booktabs}
\usepackage{latexsym}
\usepackage{hyperref}
\usepackage{cleveref}
\usepackage[export]{adjustbox}
\usepackage{dsfont}

\def\BibTeX{{\rm B\kern-.05em{\sc i\kern-.025em b}\kern-.08em
    T\kern-.1667em\lower.7ex\hbox{E}\kern-.125emX}}
\begin{document}
\title{Cross-Dataset Knowledge Transfer for \\ Traffic Scene Understanding
%via  GNN
%Heterogeneous Graph Neural Networks
}
%\title{Advanced Unsupervised Domain Adaptation for Driving Scene Graphs}

\author{
	\IEEEauthorblockN{1\IEEEauthorrefmark{1},
    2\IEEEauthorrefmark{2}, 
	3\IEEEauthorrefmark{3}, 
	4\IEEEauthorrefmark{1}, \\ 
	5\IEEEauthorrefmark{2}, and
	6\IEEEauthorrefmark{3} }
	\IEEEauthorblockA{\IEEEauthorrefmark{1}Florida International University, USA, \IEEEauthorrefmark{2}Indian Institute of Technology Kanpur, India, \\
		\IEEEauthorrefmark{3}National Yang Ming Chiao Tung University, Taiwan, 
		}
}
%\title{Traffic Scene Graphs as Universal Representations: Cross-Dataset Transfer Learning for Autonomous Driving}
% \title{Conference Paper Title*\\
% {\footnotesize \textsuperscript{*}Note: Sub-titles are not captured in Xplore and
% should not be used}
% \thanks{Identify applicable funding agency here. If none, delete this.}
% }

% \author{\IEEEauthorblockN{1\textsuperscript{st} Given Name Surname}
% \IEEEauthorblockA{\textit{dept. name of organization (of Aff.)} \\
% \textit{name of organization (of Aff.)}\\
% City, Country \\
% email address or ORCID}
% \and
% \IEEEauthorblockN{2\textsuperscript{nd} Given Name Surname}
% \IEEEauthorblockA{\textit{dept. name of organization (of Aff.)} \\
% \textit{name of organization (of Aff.)}\\
% City, Country \\
% email address or ORCID}
% \and
% \IEEEauthorblockN{3\textsuperscript{rd} Given Name Surname}
% \IEEEauthorblockA{\textit{dept. name of organization (of Aff.)} \\
% \textit{name of organization (of Aff.)}\\
% City, Country \\
% email address or ORCID}
% \and
% \IEEEauthorblockN{4\textsuperscript{th} Given Name Surname}
% \IEEEauthorblockA{\textit{dept. name of organization (of Aff.)} \\
% \textit{name of organization (of Aff.)}\\
% City, Country \\
% email address or ORCID}
% \and
% \IEEEauthorblockN{5\textsuperscript{th} Given Name Surname}
% \IEEEauthorblockA{\textit{dept. name of organization (of Aff.)} \\
% \textit{name of organization (of Aff.)}\\
% City, Country \\
% email address or ORCID}
% \and
% \IEEEauthorblockN{6\textsuperscript{th} Given Name Surname}
% \IEEEauthorblockA{\textit{dept. name of organization (of Aff.)} \\
% \textit{name of organization (of Aff.)}\\
% City, Country \\
% email address or ORCID}
% }

\maketitle

\begin{abstract}
Autonomous driving systems often struggle to generalize across diverse datasets due to varying sensor configurations, annotation quality, and environmental conditions.  
We propose a novel framework that enables universal traffic scene understanding by leveraging two key ingredients: (i) a temporal graph representing the traffic scene over time — which we term \emph{Traffic Scene Graph (TSG)} — and (ii) a knowledge-transfer mechanism that projects these graph features into a common embedding space via an \emph{Unsupervised Scene Translation (UST) network}.  
By performing inference in this common space — the \emph{Shared Graph Scene Embedding Space (SGSES)} — we can train a single network for downstream tasks that generalizes across datasets and scene types.  
%
To illustrate this approach, we focus on a risk-assessment task, transferring knowledge from a high-quality dataset (e.g., NuPlan) to a lower-quality dataset (e.g., Learn to Drive (L2D)).  
Our pipeline first extracts comprehensive features — including 3D relative velocities, depth estimates, and visual embeddings — using a multi-stage process combining object detection, depth estimation, and learned image representations.  
These features are used to build the TSG that are processed by dataset-specific encoders. A projection head trained with contrastive learning aligns the resulting embeddings in SGSES.  
%
On the risk-prediction task, our framework significantly improves L2D’s performance via knowledge transfer from NuPlan, yielding a substantial reduction in risk-assessment error compared to single-dataset baselines.  
Our results demonstrate a scalable solution to leverage multiple autonomous-driving datasets, lowering annotation requirements and enhancing generalization across diverse driving environments.
\end{abstract}


% \begin{abstract}
% Autonomous driving systems often struggle to generalize across diverse datasets due to varying sensor configurations, annotation quality, and environmental conditions. 
% %
% We propose a novel framework that enables universal traffic scene understanding by leveraging two key ingredients: (i)  a temporal graph describing the traffic scene over time-- which we term \emph{Traffic Scene Graph} (TSG)-- and (ii) a knowledge transfer mechanism which projects the graph features over a common embedding space -- the \emph{Unsupervised Scene Translation (UST) network}. 
% %
% Inference over this common embedding space -- the Shared Graph Scene Embedding Space (SGSES)-- allows one to train a common network for a common downstream tasks. 
% %
% To demonstrate the approach we consider the problem of risk assessment measures by transfer learning from a high-quality configuration to a low-quality configuration. 
% %
% In particular, the low quality setting considers the driving datasets—Learn to Drive (L2D) while the high quality one is the NuPlan and the common task is the risk assessment.
% %5
% % —into. unified graph representations, where nodes represent traffic participants and environmental conditions, while edges encode spatial-temporal relationships.
% % %
% % The core innovation lies in our \textbf{dual-encoder architecture with projection-based alignment}, which enables knowledge transfer from the feature-rich NuPlan dataset to enhance L2D predictions despite differences in sensor modalities and data quality. 
% %
% We extract comprehensive features including 3D relative velocities, depth estimates, and visual embeddings through a multi-stage pipeline combining YOLO detection, Apple Depth Pro estimation, and learned image representations. 
% %
% These features populate heterogeneous graphs processed by dataset-specific encoders, whose embeddings are aligned in a shared latent space through a projection head trained with contrastive learning.
% %
% We demonstrate our framework's effectiveness on risk prediction tasks, showing that L2D performance significantly improves through knowledge distillation from NuPlan, achieving a \% reduction in risk assessment error compared to single-dataset baselines. Our approach provides a scalable solution for leveraging multiple autonomous driving datasets, reducing annotation requirements while improving model generalization across diverse driving scenarios.
% \end{abstract}


%
\begin{IEEEkeywords}
Autonomous Driving, Graph Neural Networks, Knowledge Distillation, Multi-Modal Learning, Cross-Dataset Transfer Learning
\end{IEEEkeywords}
%

% Include the introduction from separate file
\section{Introduction}
\label{sec:intro}

\textcolor{blue}{The development of robust autonomous driving systems is fundamentally constrained by the diversity of sensor configurations and data formats across different vehicle platforms and datasets. Perception models trained on one dataset, such as the high-resolution LiDAR and camera data from nuPlan \cite{nuplan_2021}, often fail to generalize to platforms with different sensor suites, like the monocular camera-only data in the Learn to Drive (L2D) dataset \cite{l2d_2020}. This lack of interoperability creates significant barriers to progress: (i) it prevents the development of universal safety validation standards, hindering regulatory approval; (ii) it leads to fragmented, costly engineering efforts, as models must be re-designed for each new platform; and (iii) it obstructs the realization of cooperative perception in connected and autonomous vehicle (CAV) ecosystems, where shared understanding is paramount.}

To address these challenges, we propose \emph{Unsupervised Scene Translation (UST)}, a framework that maps heterogeneous scene data into a shared latent space regardless of sensor configuration or dataset origin. We introduce: (i) \emph{Traffic Scene Graphs (TSG)}—temporal graph representations capturing dynamic entities and spatio-temporal relationships, and (ii) \emph{Shared Graph Scene Embedding Space (SGSES)}—a common embedding space enabling cross-platform reasoning. We demonstrate this approach using nuPlan (high-quality) and Learn to Drive (L2D, lower-quality) datasets, evaluating on safety assessment via Time-to-Collision (TTC), Post-Encroachment Time (PET), and Deceleration Rate to Avoid Collision (DRAC) metrics.

\subsection*{Literature Review}
\noindent
\textbf{Knowledge Graphs in Traffic:} Knowledge graphs enhance traffic prediction through spatial-temporal modeling. KST-GCN and STKG-STTN architectures effectively capture correlations between external factors and spatio-temporal dependencies \cite{Zhu2020KST-GCN,Zhao2024STKG-STTN}. Visual and multimodal traffic knowledge graphs integrate heterogeneous data sources into unified representations \cite{Guo2023Visual,Tan2021Research}, advancing scene understanding capabilities.

\noindent
\textbf{Language Models for Semantics:} Large Language Models provide semantic insights from multimodal sensor data \cite{Zhang2024TransportationGames,Jain2024Semantic}. LLMs streamline ITS operations, handle multimodal traffic data, and address real-world transportation challenges \cite{Zheng2023ChatGPT,Wandelt2024Large,Zhou2023Vision}, bridging raw data and actionable intelligence.

\noindent
\textbf{Multi-Sensor Fusion:} Integrating cameras, LiDAR, and radar enhances prediction robustness. EZFusion improves 3D detection by combining complementary sensor strengths \cite{Wandelt2024Large}. CR3DT utilizes radar velocity and camera spatial data for enhanced tracking \cite{Baumann2024CR3DT}. Camera-LiDAR fusion frameworks show improvements under challenging conditions \cite{Sochaniwsky2024A}, while SparseFusion3D addresses radar sparsity \cite{Yu2024SparseFusion3D}.

\noindent
\textbf{Interpretability:} Knowledge graphs provide transparent reasoning frameworks \cite{Rajabi2022Knowledge-graph-based}. LLMs offer semantic context that, combined with knowledge graphs, improves reasoning and explainability \cite{Pan2023Unifying,Kau2024Combining}. Ontological reasoning with LLMs enhances traceability \cite{Baldazzi2024Explaining}, crucial for regulatory compliance.

\noindent
\textbf{Research Gaps:} Existing approaches lack cross-domain transfer mechanisms. Knowledge graphs focus on single datasets, LLMs remain underexplored for universal traffic representations, and fusion techniques struggle with generalization across sensor suites. A unified framework for learning universal traffic scene representations across diverse datasets is absent.

\subsection*{Contributions}
\begin{itemize}[leftmargin=*]
	\item \textcolor{blue}{\textbf{Universal Traffic Scene Graph (TSG) Representation:} We propose a novel temporal semantic graph structure that serves as a universal, sensor-agnostic representation for traffic scenes. This abstraction layer enables consistent interpretation of data from vastly different sources.}
	\item \textcolor{blue}{\textbf{Unsupervised Scene Translation (UST):} We introduce a novel unsupervised alignment mechanism that maps heterogeneous TSGs into a Shared Graph Scene Embedding Space (SGSES). This enables knowledge transfer between datasets without requiring paired data, a significant advantage in real-world applications.}
	\item \textcolor{blue}{\textbf{End-to-End Multi-Modal Feature Extraction:} We present a comprehensive pipeline for constructing robust scene graphs. This pipeline integrates object detection, monocular depth estimation, 3D velocity computation from 2D data, and learned visual embeddings, overcoming the limitations of datasets with sparse annotations.}
	\item \textcolor{blue}{\textbf{Heterogeneous Graph Neural Network (GNN) Architecture:} We design a specialized heterogeneous GNN that learns consistent cross-dataset embeddings while preserving the unique, task-specific information inherent in each dataset. This is achieved through a dual-encoder architecture with a shared projection head.}
	\item \textcolor{blue}{\textbf{Empirical Validation and Scalability:} We demonstrate the effectiveness of our framework through extensive experiments on a risk assessment task. By transferring knowledge from the nuPlan to the L2D dataset, we show a significant reduction in risk assessment error, highlighting the potential for our approach to enhance safety and reduce the need for dataset-specific engineering.}
\end{itemize}

\subsection*{Reproducibility}
Our implementation is publicly available at: \url{https://github.com/mtalonso-research/Traffic_Semantic_Graphs}.


\begin{figure*}[!t]
\centering
\includegraphics[width=\textwidth]{figures/project_overview.pdf}
\caption{System architecture for cross-dataset knowledge distillation. The pipeline processes raw sensor data from NuPlan and Learn to Drive (L2D) datasets through feature extraction, constructs heterogeneous traffic scene graphs, and employs dual encoders with a shared projection head for knowledge transfer. The framework jointly optimizes feature reconstruction, link prediction, and KL divergence losses to learn universal traffic representations for downstream risk prediction.}
\label{fig:system_architecture}
\end{figure*}



% \section{System Model}
% \label{sec:system_model}

% We consider two (or more) autonomous-driving datasets, denoted as \(\mathcal{D}_1\) and \(\mathcal{D}_2\).  
% Each dataset \(\mathcal{D}_j\) consists of \(N_j\) episodes  
% \[
% \mathcal{D}_j = \{E^j_1, E^j_2, \dots, E^j_{N_j}\},
% \]  
% where each episode \(E^j_i\) represents a temporal recording from a vehicle, containing a sequence of sensor observations, e.i. 
% \[
% \mathcal{E}^j_i = \{o^j_{i,1}, o^j_{i,2}, \dots, o^j_{i,T^j_i}\}.
% \]  
% Each observation \(o^j_{i,t}\) may contain data from multiple sensors (e.g., camera images, LiDAR point clouds, radar returns), depending on the sensor configuration of the dataset. 
% %
% We assume that \(\mathcal{D}_1\) and \(\mathcal{D}_2\) may differ substantially in their sensor modalities, annotation granularity, sampling rates, and overall data quality.
% %
% Additionally, for each episode \(E^j_i\), we assume a (possibly noisy) scalar risk measure \(r^j_i\) is available — encoding e.g., a risk score, safety metric, or surrogate indicator. 
% %
% We note that the noise level or reliability of this risk measure may differ across datasets: one dataset may provide high-fidelity annotations or richer sensor data leading to more accurate risk estimation, while another dataset may contain noisier or less reliable risk labels.

% Our goal is to learn a unified, data-driven latent representation space  
% \[
% \mathcal{Z} \subset \mathbb{R}^d,
% \]  
% via a dataset-specific mapping  
% \[
% f_j: \mathcal{E}^j_i \;\rightarrow\; \mathcal{Z},
% \]  
% such that observations from either dataset — regardless of their sensor modalities or quality — can be projected into \(\mathcal{Z}\).  

% In this common latent space \(\mathcal{Z}\), we then train a prediction model  
% \[
% g: \mathcal{Z} \;\rightarrow\; \hat r,
% \]  
% that maps latent embeddings to risk estimates. By leveraging the cleaner (less noisy) risk labels from dataset \(\mathcal{D}_1\) during training of \(g\), and using the shared embedding \(f\) to project \(\mathcal{D}_2\) data into \(\mathcal{Z}\), we aim to improve the risk prediction accuracy on \(\mathcal{D}_2\). In other words, we perform **transfer learning from the reliable dataset \(\mathcal{D}_1\) to the noisier dataset \(\mathcal{D}_2\)**.

% Formally, during training we optimize:

% \[
% \min_{f, g}\; \mathcal{L}_\text{risk}\bigl(g(f(\mathcal{O}^1_i)), r^1_i\bigr) \;+\; \lambda\, \mathcal{L}_\text{align}\bigl(f(\mathcal{O}^1), f(\mathcal{O}^2)\bigr),
% \]

% where \(\mathcal{L}_\text{risk}\) is a supervised loss on risk prediction using the high-quality dataset \(\mathcal{D}_1\), and \(\mathcal{L}_\text{align}\) is an unsupervised alignment loss (e.g., contrastive or distribution-matching) that encourages embeddings from \(\mathcal{D}_1\) and \(\mathcal{D}_2\) to share the same latent space distribution.

% At test time, for episodes from \(\mathcal{D}_2\), we compute risk estimates as \(\hat r_i = g(f(\mathcal{O}^2_i))\), thereby realizing cross-dataset risk prediction despite differences in sensor setup and annotation quality.

\section{System Model}
\label{sec:system_model}

We consider two (or more) autonomous-driving datasets, denoted as \(\mathcal{D}_1\) and \(\mathcal{D}_2\).  
Each dataset \(\mathcal{D}_j\) consists of \(N_j\) episodes  
\[
\mathcal{D}_j = \{E^j_1, E^j_2, \dots, E^j_{N_j}\},
\]  
where each episode \(E^j_i\) represents a temporal recording from a vehicle, containing a sequence of sensor observations  
\[
\mathcal{O}^j_i = \{o^j_{i,1}, o^j_{i,2}, \dots, o^j_{i,T^j_i}\}.
\]  
Each observation \(o^j_{i,t}\) may contain data from multiple sensors (e.g., camera images, LiDAR point clouds, radar returns), depending on the sensor configuration of the dataset.  

We assume that \(\mathcal{D}_1\) and \(\mathcal{D}_2\) may differ substantially in their sensor modalities, annotation granularity, sampling rates, and overall data quality.  

For each episode \(E^j_i\), we assume a (possibly noisy) scalar risk measure \(r^j_i\) is available — encoding e.g., a risk score, safety metric, or surrogate indicator. We model the label noise as additive:  
\[
r^j_i = r^*_i + \varepsilon^j_i,\quad \varepsilon^j_i \sim \mathcal{N}\bigl(0, \sigma_j^2\bigr),
\]  
where \(r^*_i\) is the (unobserved) true underlying risk of the episode, and \(\sigma_j^2\) captures the noise level (uncertainty) of dataset \(\mathcal{D}_j\). We assume \(\sigma_1^2 < \sigma_2^2\), i.e., \(\mathcal{D}_1\) has more reliable (less noisy) risk labels than \(\mathcal{D}_2\).  

Our goal is to learn a unified, data-driven latent representation space  
\[
\mathcal{Z} \subset \mathbb{R}^d,
\]  
via dataset-specific encoders  
\[
f_j: \mathcal{O}^j_i \;\rightarrow\; \mathcal{Z},
\]  
such that observations from either dataset — regardless of their sensor modalities or quality — can be projected into \(\mathcal{Z}\).  

In this common latent space, we then train a shared risk predictor  
\[
g: \mathcal{Z} \;\rightarrow\; \hat r,
\]  
so that \(\hat r = g\bigl(f_j(\mathcal{O}^j_i)\bigr)\).  

During training, we primarily leverage the cleaner dataset \(\mathcal{D}_1\) to learn \(g\), while using an alignment loss to bring the embeddings of \(\mathcal{D}_2\) into the same latent space as \(\mathcal{D}_1\). Formally, we optimize  

\[
\min_{f_1, f_2, g}\; \frac{1}{N_1}\sum_{i=1}^{N_1} \ell\bigl(g(f_1(\mathcal{O}^1_i)), r^1_i\bigr) \;+\; \lambda\, \mathcal{L}_\mathrm{align}\bigl(\{f_1(\mathcal{O}^1)\}, \{f_2(\mathcal{O}^2)\}\bigr),
\]  

where \(\ell\) is a supervised loss (e.g. squared error), and \(\mathcal{L}_\mathrm{align}\) is an unsupervised distribution-or contrastive-based alignment loss encouraging the two embedding distributions to match.  

At test time, for episodes from \(\mathcal{D}_2\), we compute embeddings via \(f_2\) and risk predictions \(\hat r_i = g(f_2(\mathcal{O}^2_i))\), obtaining risk estimates for the noisier dataset without having used its noisy labels for training the predictor.  

\section{Proposed Approach}


\vspace{5cm}
% Consider two autonomous driving datasets $\mathcal{D}_1$ (NuPlan) and $\mathcal{D}_2$ (L2D), each comprising $N$ episodes $\mathcal{E} = \{e_1, e_2, ..., e_n\}$ where each episode $e_i$ contains a sequence of $T$ temporal observations $\mathcal{O}_i = \{o_1^i, o_2^i, ..., o_T^i\}$. 
% %
% The datasets exhibit fundamental differences in sensor modalities, annotation granularity, and data quality. 
% %


% Our objective is to learn a unified representation function $f: \mathcal{O} \rightarrow \mathcal{Z}$ that maps observations from either dataset to a shared latent space $\mathcal{Z} \in \mathbb{R}^d$, enabling knowledge transfer from the annotation-rich dataset $\mathcal{D}_1$ to enhance predictions on $\mathcal{D}_2$.




% \vspace{3cm}
\subsection{Multi-Modal Feature Extraction Pipeline}

\subsubsection{Temporal Sampling Strategy}

Given the computational and storage constraints inherent in processing high-frequency sensor data, we implement a strategic temporal sampling approach. For an episode with duration $T_{episode}$, we extract observations at intervals $\Delta t = 3$ seconds, yielding a sequence:
\begin{equation}
\mathcal{O}_{sampled} = \{o_t : t \in \{0, \Delta t, 2\Delta t, ..., \lfloor T_{episode}/\Delta t \rfloor \cdot \Delta t\}\}
\end{equation}

This sampling rate balances temporal resolution with computational tractability, though it introduces challenges in maintaining object persistence across frames, particularly for dynamic entities with high relative velocities.

\subsubsection{Environmental Context Enrichment}

We augment native dataset annotations with external data sources to establish comparable environmental representations across datasets.

For the NuPlan dataset, we perform coordinate system transformations from local East-North-Up (ENU) frames to global WGS84 coordinates. Given a regional origin $(lat_0, lon_0, alt_0)$, we apply the transformation:
\begin{equation}
[lon, lat, alt]^T = \Pi_{ENU \rightarrow WGS84}([x, y, z]^T; lat_0, lon_0)
\end{equation}
where $\Pi$ represents the projection transformation using the transverse Mercator projection centered at the regional origin.

Subsequently, we query the Open-Meteo Archive API to obtain hourly weather data, matching timestamps with microsecond precision. The weather feature vector encompasses:
\begin{equation}
\mathbf{w}_{weather} = \{\rho_{precip}, \omega_{code}, \delta_{daylight}\}
\end{equation}
where $\rho_{precip}$ denotes precipitation in millimeters, $\omega_{code}$ represents WMO weather codes, and $\delta_{daylight}$ is a binary daylight indicator.

For the L2D dataset, which lacks comprehensive map annotations, we leverage the OpenStreetMap Overpass API to extract traffic infrastructure within a 30-meter radius of each observation:
\begin{equation}
\mathcal{M}_{OSM} = \text{Query}_{Overpass}(lat \pm \epsilon, lon \pm \epsilon), \quad \epsilon = 30m
\end{equation}

The extracted features include traffic controls $\mathcal{C} = \{$stop\_sign, traffic\_signal, yield\_sign$\}$, road attributes $\mathcal{R} = \{$lanes, maxspeed, surface\_type$\}$, and traffic features $\mathcal{F} = \{$pedestrian\_crossing, bus\_stop$\}$.

\subsubsection{3D Velocity Estimation from Monocular Vision}

For the L2D dataset, which provides only RGB imagery, we implement a multi-stage computer vision pipeline to reconstruct 3D motion dynamics. Given consecutive frames $I^t$ and $I^{t+\Delta t}$, we compute:

1. **Object Detection**: Apply YOLO-based detection to identify traffic participants:
   \begin{equation}
   \mathcal{B}^t = \text{YOLO}(I^t) = \{b_i^t = (u_i, v_i, w_i, h_i, c_i)\}_{i=1}^{N}
   \end{equation}
   where $(u_i, v_i)$ denotes bounding box center, $(w_i, h_i)$ represents dimensions, and $c_i$ is the class confidence.

2. **Depth Estimation**: Employ Apple Depth Pro for monocular depth prediction:
   \begin{equation}
   D^t = \Phi_{depth}(I^t): \mathbb{R}^{H \times W \times 3} \rightarrow \mathbb{R}^{H \times W}
   \end{equation}

3. **Relative Velocity Computation**: Calculate 3D velocities using temporal depth variations and pixel displacements:
   \begin{equation}
   v_z^{rel} = \frac{1}{\Delta t}[D^{t+\Delta t}(u_i, v_i) - D^t(u_i, v_i)]
   \end{equation}
   \begin{equation}
   [v_x^{rel}, v_y^{rel}]^T = \frac{\bar{D}^t}{\Delta t \cdot f} \mathbf{K}^{-1} [\Delta u_i, \Delta v_i, 1]^T
   \end{equation}
   where $\mathbf{K}$ represents the camera intrinsic matrix, $f$ denotes focal length, and $\bar{D}^t$ is the mean depth within the bounding box.

\subsection{Semantic Scene Understanding}

\subsubsection{Behavioral Classification}

We implement a hierarchical classification scheme for ego vehicle maneuvers based on steering angle integration and turn signal states. The cumulative steering angle over a temporal window is computed as:
\begin{equation}
\Phi_{cum} = \sum_{t=t_0}^{t_0+w} \phi_t \cdot \Delta t
\end{equation}

The behavioral classification follows:
\begin{equation}
\mathcal{B}_{ego} = \begin{cases}
    \text{lane\_change} & \text{if } |\Phi_{cum}| \in [3°, 30°] \wedge \tau \neq 0 \\
    \text{turning} & \text{if } |\Phi_{cum}| \in (30°, 135°] \\
    \text{u-turn} & \text{if } |\Phi_{cum}| > 135° \\
    \text{straight} & \text{otherwise}
\end{cases}
\end{equation}
where $\tau \in \{0, 1, 2\}$ represents the turn signal state (none, left, right).

\subsubsection{Hierarchical Semantic Tagging}

We assign multi-dimensional semantic tags to characterize each driving episode across four primary categories:

1. **Action Tags** $\mathcal{T}_{action}$: Dominant maneuver classification with priority ordering for complex behaviors (roundabout $>$ parking $>$ turning $>$ straight)

2. **Traffic Control Tags** $\mathcal{T}_{control}$: Regulatory elements with hierarchical precedence (signal $>$ stop $>$ yield $>$ unmarked)

3. **Road Feature Tags** $\mathcal{T}_{road}$: Binary indicators for infrastructure characteristics including speed zones, road types, and special structures

4. **Environmental Tags** $\mathcal{T}_{env}$: Temporal and visibility conditions including rush hour detection (07:00-09:00, 16:00-18:00 on weekdays) and seasonal indicators

\subsection{Heterogeneous Graph Representation}

We model each traffic scene as a heterogeneous directed graph $\mathcal{G}^t = (\mathcal{V}^t, \mathcal{E}^t, \mathcal{X}^t, \mathcal{R})$ where $\mathcal{V}^t$ represents typed nodes, $\mathcal{E}^t$ denotes edges, $\mathcal{X}^t$ contains feature matrices, and $\mathcal{R}$ defines relation types.

\subsubsection{Node Architecture}

The node set comprises four distinct types:
\begin{equation}
\mathcal{V}^t = \mathcal{V}_{ego}^t \cup \mathcal{V}_{vehicle}^t \cup \mathcal{V}_{pedestrian}^t \cup \mathcal{V}_{environment}^t
\end{equation}

Each node type maintains a specific feature schema:
- **Ego node**: $\mathbf{x}_{ego} \in \mathbb{R}^{d_{ego}}$ encoding position, velocity, acceleration, heading, and visual embeddings
- **Vehicle nodes**: $\mathbf{x}_{veh} \in \mathbb{R}^{d_{veh}}$ containing relative position, velocity, and distance metrics
- **Pedestrian nodes**: $\mathbf{x}_{ped} \in \mathbb{R}^{d_{ped}}$ with motion and proximity features
- **Environment node**: $\mathbf{x}_{env} \in \mathbb{R}^{d_{env}}$ aggregating weather, temporal, and infrastructure context

\subsubsection{Edge Construction Strategy}

We establish typed edges based on spatial-temporal relationships:
\begin{equation}
\mathcal{E}^t = \bigcup_{r \in \mathcal{R}} \mathcal{E}_r^t
\end{equation}

where relation types include:
- **Proximity relations**: Connect entities when $d_{ij} < \theta_{dist}$ or $\text{TTC}_{ij} < \theta_{time}$
- **Context relations**: Link all nodes to the environment node for global context propagation
- **Temporal relations**: Connect same entities across consecutive frames for motion consistency

\subsection{Knowledge Distillation Architecture}

\subsubsection{Dataset-Specific Encoders}

We employ separate Graph Neural Network encoders tailored to each dataset's characteristics. For dataset $d \in \{$NuPlan, L2D$\}$, the encoder performs:
\begin{equation}
\mathbf{H}_d^{(l+1)} = \text{HeteroGNN}(\mathbf{H}_d^{(l)}, \mathcal{G}_d^t; \Theta_d^{(l)})
\end{equation}

Each layer aggregates information through typed message passing:
\begin{equation}
\mathbf{h}_{v}^{(l+1)} = \sigma\left(\mathbf{W}_{self}^{(l)} \mathbf{h}_{v}^{(l)} + \sum_{r \in \mathcal{R}} \sum_{u \in \mathcal{N}_r(v)} \alpha_{r,uv} \mathbf{W}_r^{(l)} \mathbf{h}_u^{(l)}\right)
\end{equation}
where $\alpha_{r,uv}$ represents attention weights computed via:
\begin{equation}
\alpha_{r,uv} = \frac{\exp(\text{LeakyReLU}(\mathbf{a}_r^T[\mathbf{W}_r \mathbf{h}_u || \mathbf{W}_r \mathbf{h}_v]))}{\sum_{u' \in \mathcal{N}_r(v)} \exp(\text{LeakyReLU}(\mathbf{a}_r^T[\mathbf{W}_r \mathbf{h}_{u'} || \mathbf{W}_r \mathbf{h}_v]))}
\end{equation}

\subsubsection{Projection and Alignment}

A shared projection head maps dataset-specific embeddings to a unified latent space:
\begin{equation}
\mathbf{z}_d = \psi(\text{READOUT}(\mathbf{H}_d^{(L)}); \Omega)
\end{equation}
where $\psi$ is a multi-layer perceptron with shared parameters $\Omega$, and READOUT performs hierarchical pooling across node types.

\subsection{Multi-Objective Optimization}

The training objective combines reconstruction, structural preservation, and distributional alignment:

\begin{equation}
\mathcal{L} = \sum_{d \in \{1,2\}} \lambda_{recon}^d \mathcal{L}_{recon}^d + \lambda_{link}^d \mathcal{L}_{link}^d + \lambda_{align} \mathcal{L}_{align} + \lambda_{task} \mathcal{L}_{task}
\end{equation}

where:
- $\mathcal{L}_{recon}^d$: Feature reconstruction via graph decoders
- $\mathcal{L}_{link}^d$: Link prediction for structural preservation
- $\mathcal{L}_{align}$: KL divergence between projected distributions
- $\mathcal{L}_{task}$: Downstream risk prediction loss

The framework employs curriculum learning, initially emphasizing reconstruction objectives before gradually increasing the weight of alignment and task-specific losses as training progresses.


\section{Methodology}
\label{sec:methodology}
\section{Experimental Setup}
\label{sec:experiments}
\section{Results}
\label{sec:results}
\section{Analysis and Discussion}
\label{sec:analysis}
\section{Conclusion}
\label{sec:conclusion}


% References should be produced using the bibtex program from suitable
% BiBTeX files (here: strings, refs, manuals). The IEEEbib.bst bibliography
% style file from IEEE produces unsorted bibliography list.
% -------------------------------------------------------------------------
\bibliographystyle{IEEEtran}
\bibliography{IEEEabrv,KLASS.bib}

\end{document}